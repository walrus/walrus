\documentclass[a4paper]{article}

% Table of contents depth
\setcounter{tocdepth}{3}

% Section numbering depth (zero for no numbering)
\setcounter{secnumdepth}{0}

% LaTeX package inclusions
\usepackage[english]{babel}

\usepackage{fullpage} % Page width options

\usepackage{hyperref} % Internal and external references
\usepackage{url}      % Unused?
\usepackage{breakurl} % Support for sensible line breaks in URLS

\usepackage{tabulary} % Fun with tables
\usepackage{float}    % Allows for better placement of tables etc
\usepackage{array}    % More options for tables
\usepackage{multirow} % Support for multi row columns in tables

\usepackage{graphicx} % For picture inclusion
\graphicspath{{images/}}

\usepackage[colorinlistoftodos]{todonotes} % For the inclusion of TODOs
\usepackage[toc,page]{appendix} % Generation of bibliography/appendix

% Source code inclusion
\usepackage{listings}
\lstset{
  tabsize=2,
  basicstyle = \ttfamily\small,
  columns=fullflexible
}
% Usage for the above like so:
% \begin{lstlisting}
%   CODE CODE CODE
% \end{lstlisting}

% In-line code styling (same style as listing)
\newcommand{\shell}[1]{\lstinline{#1}}

% Use roman numerals for page numbers in the contents
\pagenumbering{roman}

%%%%%%%%%%%%%%%%%%%%%%%%%%%%%%%%%%%%%%%%%%%%%%%%%%%%%%%%%%%%%%%%%%%%%%%%%%
%%%%%%%%%%%%%%%%%%%%%%%%%%%%%%%%%%%%%%%%%%%%%%%%%%%%%%%%%%%%%%%%%%%%%%%%%%
%%%%%%%%%%%%%%%%%%%%%%%%%%%%%%%%%%%%%%%%%%%%%%%%%%%%%%%%%%%%%%%%%%%%%%%%%%

\begin{document}

\title{\textbf{WALRUS:}\\
Exercise classification on embedded hardware}
\date{2017}
\author{
Daniel Clay \\
\emph{Supervised by Dr Thomas Heinis} 
}
\maketitle
\pagebreak
%%%%%%%%%%%%%%%%%%%%%%%%%%%%%%%%%%%%%%%%%%%%%%%%%%%%%%%%%%%%%%%%%%%%%%%%%%

%%%%%%%%%%%%%%%%%%%%%%%%%%%%%%%%%%%%%%%%%%%%%%%%%%%%%%%%%%%%%%%%%%%%%%%%%%
\section{Abstract}
%%%%%%%%%%%%%%%%%%%%%%%%%%%%%%%%%%%%%%%%%%%%%%%%%%%%%%%%%%%%%%%%%%%%%%%%%%

\todo[inline]{Abstract: do this last, goes the advice}

%%%%%%%%%%%%%%%%%%%%%%%%%%%%%%%%%%%%%%%%%%%%%%%%%%%%%%%%%%%%%%%%%%%%%%%%%%
\section{Acknowledgements}
%%%%%%%%%%%%%%%%%%%%%%%%%%%%%%%%%%%%%%%%%%%%%%%%%%%%%%%%%%%%%%%%%%%%%%%%%%

\todo[inline]{Acknowledgements: Dr Heinis et al}

%%%%%%%%%%%%%%%%%%%%%%%%%%%%%%%%%%%%%%%%%%%%%%%%%%%%%%%%%%%%%%%%%%%%%%%%%%

% Table of Contents on new page
\pagebreak
\tableofcontents
\pagebreak

% Use arabic numerals after contents
\pagenumbering{arabic}

%%%%%%%%%%%%%%%%%%%%%%%%%%%%%%%%%%%%%%%%%%%%%%%%%%%%%%%%%%%%%%%%%%%%%%%%%%
\section{Introduction}
%%%%%%%%%%%%%%%%%%%%%%%%%%%%%%%%%%%%%%%%%%%%%%%%%%%%%%%%%%%%%%%%%%%%%%%%%%

\subsection{Motivation}%%%%%%%%%%%%%%%%%%%%%%%%%

\todo[inline]{I have some, probably}

\subsection{Issues}%%%%%%%%%%%%%%%%%%%%%%%%%

\todo[inline]{What makes this difficult}

\subsection{Contributions}%%%%%%%%%%%%%%%%%%%%%%%%%%%%%%%%%%%%%%%%%%%%%%%%

\todo[inline]{As it says on the tin. List all the deliverables}

%%%%%%%%%%%%%%%%%%%%%%%%%%%%%%%%%%%%%%%%%%%%%%%%%%%%%%%%%%%%%%%%%%%%%%%%%%
\section{Background}
%%%%%%%%%%%%%%%%%%%%%%%%%%%%%%%%%%%%%%%%%%%%%%%%%%%%%%%%%%%%%%%%%%%%%%%%%%

\subsection{Neural Networks}%%%%%%%%%%%%%%%%%%%%%%%%%%%%%%%%%%%%%%%%%%%%%%

\todo[inline]{What is the current state of research regarding neural networks?, mostly from Interim Report. Also discuss what frameworks are available?}

\subsection{Embedded Systems}%%%%%%%%%%%%%%%%%%%%%%%%%%%%%%%%%%%%%%%%%%%%%

\todo[inline]{Current state of embedded systems etc, including performance}

\subsection{Bluetooth / Bluetooth Low Energy (BLE)}%%%%%%%%%%%%%%%%%%%%%%%

\todo[inline]{Look, it's a thing. Explain why it's useful}

\subsection{Fitness Trackers}%%%%%%%%%%%%%%%%%%%%%%%%%%%%%%%%%%%%%%%%%%%%%

\todo[inline]{Current state of fitness trackers available, mostly from Interim Report}

\subsection{Intel Curie}%%%%%%%%%%%%%%%%%%%%%%%%%

\todo[inline]{Info about the Intel Curie. Note what is currently available, and what could be available soon, mostly from Interim Report}

\subsection{Bodyweight Exercises}%%%%%%%%%%%%%%%%%%%%%%%%%

\todo[inline]{Discuss these exercises and their form, mostly from Interim Report}

%%%%%%%%%%%%%%%%%%%%%%%%%%%%%%%%%%%%%%%%%%%%%%%%%%%%%%%%%%%%%%%%%%%%%%%%%%
\section{Design \& Specification}
%%%%%%%%%%%%%%%%%%%%%%%%%%%%%%%%%%%%%%%%%%%%%%%%%%%%%%%%%%%%%%%%%%%%%%%%%%

\todo[inline]{Overall design. Note different versions, and throughout note differences due to different versions. This is basically a specification}

\subsection{Hardware}%%%%%%%%%%%%%%%%%%%%%%%%%%%%%%%%%%%%%%%%%%%%%%%%%%%%%

\todo[inline]{Overview of hardware used and specifications for alternative}

\subsubsection{Microcontroller}

\todo[inline]{Details of the microcontroller and specification. Also touch on alternatives. Also also pick a consistent name}

\subsubsection{Linux Computer}

\todo[inline]{For training. Details of my machine and specification}

\subsubsection{Android Phone}

\todo[inline]{Details of my phone and specifications needed}

\subsection{Microcontroller Software Architecture}%%%%%%%%%%%%%%%%%%%%%%%%

\todo[inline]{Software architecture \& UML diagrams etc for microcontroller}

\subsection{Linux Computer Software Architecture}%%%%%%%%%%%%%%%%%%%%%%%%%

\todo[inline]{Software architecture \& UML diagrams etc for computer}

\subsection{Android Phone Software Architecture}%%%%%%%%%%%%%%%%%%%%%%%%%%

\todo[inline]{Software architecture \& UML diagrams etc for Android phone}

\subsection{Data Capture \& Format}%%%%%%%%%%%%%%%%%%%%%%%%%%%%%%%%%%%%%%%%

\todo[inline]{Data capture and *format specification*. Talk about Curie's IMU, and how to turn the stream of inputs into the desired format for the Neural network}

\subsubsection{Log File Specification (Supervised)}

A supervised log file will be a \lstinline{.txt} file, and generally be called \lstinline{set}, followed by a numerical suffix, for example \lstinline{set3.txt}. It is recommended that log files be stored in folders such that it is obvious whose exercises have been recorded.

The format of the file will be as follows; zero or more instances of \emph{repetitions}, each consisting of:

\begin{itemize}
\item One or more lines with the values of the inputs.
\item A line with the text \lstinline{Repetition Start}
\item One or more lines with the values of the inputs.
\item A line with the text \lstinline{Repetition End}
\item One or more lines with the values of the inputs.
\end{itemize}

An example of a valid supervised log file is as follows (this example contains two valid repetitions):

\begin{lstlisting}
16003
15892
14374
14338
15569
Repetition Start
16692
22893
23160
24567
21970
26383
21008
22793
21024
20603
18865
18607
16515
15113
16346
14239
18071
18354
18466
Repetition End
17255
16897
16973
16911
18007
16843
16108
13672
13464
16149
16609
17538
Repetition Start
22227
22166
22201
21680
19268
20274
23043
20271
19949
18652
18440
18251
16334
13850
13632
12386
Repetition End
17390
16154
17653
18861
\end{lstlisting}

\subsubsection{Log File Specification (Unsupervised)}

An unsupervised log file will be a \lstinline{.txt} file, and generally be called \lstinline{set}, followed by a numerical suffix, for example \lstinline{set3.txt}. It is recommended that log files be stored in folders such that it is obvious whose exercises have been recorded.

The format of the file will be as follows; zero or more instances of \emph{repetitions}, each consisting of:

\begin{itemize}
\item A line with the text \lstinline{Motion detected after X milliseconds. Logging...}
\item One or more lines with the values of the inputs.
\item A line with the text \lstinline{Motion ended after Y milliseconds. Logging...}
\end{itemize}

Where X and Y are the intervals between the last motion detection event.

An example of a valid unsupervised log file is as follows (this example contains one valid repetition and an unfinished one, which will be removed in the normalisation process):

\begin{lstlisting}
Motion detected after  823  milliseconds. Logging...
18034
17662
22144
19415
26548
25319
26721
26086
26633
24458
25198
20258
24702
21414
19088
20752
18575
18684
Motion ended after  1718  milliseconds. Logging...
Motion detected after  892  milliseconds. Logging...
18376
18623
22552
22596
23354
23849
23313
20727
23364
\end{lstlisting}


\subsubsection{Normalised Log File Specification}

A normalised log file will be a \lstinline{.txt} file, and have the name of the set from which it was normalised, followed by the suffix \lstinline{_normalised}. For example, the normalised log file for the (raw) log file \lstinline{set1.txt} would be \lstinline{set1_normalised.txt}.

The format of the file will be as follows; zero or more instances of \emph{repetitions}, each consisting of:
\begin{itemize}
\item A line with the text \lstinline{Repetition Start}
\item One or more lines with the values of the inputs, with one line/value per input node.
\item A line with the text \lstinline{Repetition End}
\item One or more lines with the values of the targets, with one line/value per output node.
\end{itemize}

There should be no blank lines between repetitions.

An example of a valid repetition for a network with 20 input nodes and one output node is as follows:

\begin{lstlisting}
Repetition start
20119
20119
23526
25971
30233
33248
33248
34300
29982
29982
27556
25647
22664
17422
17422
17422
15862
15862
16160
18608
Repetition end
1
\end{lstlisting}

\subsection{Neural Network Architecture}%%%%%%%%%%%%%%%%%%%%%%%%%%%%%%%%%%

\todo[inline]{Architecture of the various neural networks. Note that the parameters etc will be decided on after experimentation}

\subsection{Training of Neural Network}%%%%%%%%%%%%%%%%%%%%%%%%%%%%%%%%%%%

\todo[inline]{How to train neural network on computer (and to get the network weights etc back onto the Curie}

\subsection{Classification by the Neural Network}%%%%%%%%%%%%%%%%%%%%%%%%%

\todo[inline]{Spec for how neural network should classify examples, and output data format for display}

\subsection{Connectivity}%%%%%%%%%%%%%%%%%%%%%%%%%%%%%%%%%%%%%%%%%%%%%%%%%

\todo[inline]{Data exchange protocol specification etc. How do connect between microcontroller and phone???}

\subsection{Data Display}%%%%%%%%%%%%%%%%%%%%%%%%%%%%%%%%%%%%%%%%%%%%%%%%%

\todo[inline]{What the phone should display in response to different things}

%%%%%%%%%%%%%%%%%%%%%%%%%%%%%%%%%%%%%%%%%%%%%%%%%%%%%%%%%%%%%%%%%%%%%%%%%%
\section{Project Plan and Management}
%%%%%%%%%%%%%%%%%%%%%%%%%%%%%%%%%%%%%%%%%%%%%%%%%%%%%%%%%%%%%%%%%%%%%%%%%%

\subsection{Timetable}%%%%%%%%%%%%%%%%%%%%%%%%%%%%%%%%%%%%%%%%%%%%%%%%%%%%

\todo[inline]{Subsubsections.}

\subsection{Version Control}%%%%%%%%%%%%%%%%%%%%%%%%%%%%%%%%%%%%%%%%%%%%%%

\todo[inline]{Subsubsections.}

\subsection{Methodologies Used}%%%%%%%%%%%%%%%%%%%%%%%%%%%%%%%%%%%%%%%%%%%

\todo[inline]{Subsubsections. Include this at all?}

\subsection{Languages \& Libraries Used}%%%%%%%%%%%%%%%%%%%%%%%%%%%%%%%%%%

\todo[inline]{Subsubsections.}

\subsection{Tools Used}%%%%%%%%%%%%%%%%%%%%%%%%%%%%%%%%%%%%%%%%%%%%%%%%%%%

\todo[inline]{Subsubsections.}

\todo[inline]{More Subsections? (Hardware?}

%%%%%%%%%%%%%%%%%%%%%%%%%%%%%%%%%%%%%%%%%%%%%%%%%%%%%%%%%%%%%%%%%%%%%%%%%%
\section{Evaluation}
%%%%%%%%%%%%%%%%%%%%%%%%%%%%%%%%%%%%%%%%%%%%%%%%%%%%%%%%%%%%%%%%%%%%%%%%%%

\subsection{Contribution/Deliverable Progress \& Completion}%%%%%%%%%%%%%%

\todo[inline]{Better name? Which deliverables did I actually deliver}

\subsection{Classifier Performance}%%%%%%%%%%%%%%%%%%%%%%%%%%%%%%%%%%%%%%%

\todo[inline]{All the normal classification measures. Will probably need numbers for all the versions}

\subsection{Energy Consumption}%%%%%%%%%%%%%%%%%%%%%%%%%%%%%%%%%%%%%%%%%%%

\todo[inline]{How long did the battery last? Is this practicable for the smaller form factor Curie etc}

\subsection{Stability \& Ease of Use}%%%%%%%%%%%%%%%%%%%%%%%%%

\todo[inline]{Does it crash? Is it easy to install and use?}

%%%%%%%%%%%%%%%%%%%%%%%%%%%%%%%%%%%%%%%%%%%%%%%%%%%%%%%%%%%%%%%%%%%%%%%%%%
\section{Conclusions and Future Work}
%%%%%%%%%%%%%%%%%%%%%%%%%%%%%%%%%%%%%%%%%%%%%%%%%%%%%%%%%%%%%%%%%%%%%%%%%%

\subsection{Conclusion}%%%%%%%%%%%%%%%%%%%%%%%%%%%%%%%%%%%%%%%%%%%%%%%%%%%

\todo[inline]{What have *we* learned? What worked and what didn't, etc}

\subsection{Future Work}%%%%%%%%%%%%%%%%%%%%%%%%%%%%%%%%%%%%%%%%%%%%%%%%%%

\todo[inline]{Where can the technology go from here?}

%%%%%%%%%%%%%%%%%%%%%%%%%%%%%%%%%%%%%%%%%%%%%%%%%%%%%%%%%%%%%%%%%%%%%%%%%%
\section{Appendix 1: User Guide}
%%%%%%%%%%%%%%%%%%%%%%%%%%%%%%%%%%%%%%%%%%%%%%%%%%%%%%%%%%%%%%%%%%%%%%%%%%

\subsection{Installation}%%%%%%%%%%%%%%%%%%%%%%%%%%%%%%%%%%%%%%%%%%%%%%%%%

\todo[inline]{How to: install all the stuff. Ideally git repo + single script to install dependencies should be enough}

\subsection{Setup}%%%%%%%%%%%%%%%%%%%%%%%%%%%%%%%%%%%%%%%%%%%%%%%%%%%%%%%%

\todo[inline]{How to get ready to classify (inc. training the network to match your movements, probably. If not this section becomes less useful).}

\subsection{Usage}%%%%%%%%%%%%%%%%%%%%%%%%%%%%%%%%%%%%%%%%%%%%%%%%%%%%%%%%

\todo[inline]{How to: actually use the thing}

%%%%%%%%%%%%%%%%%%%%%%%%%%%%%%%%%%%%%%%%%%%%%%%%%%%%%%%%%%%%%%%%%%%%%%%%%%
% Bibliography
\newpage
\addcontentsline{toc}{section}{Bibliography}

\begin{thebibliography}{9}

%\bibitem{yes}

\end{thebibliography}

\end{document}
